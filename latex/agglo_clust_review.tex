\documentclass[10pt,twocolumn,letterpaper]{article}

\usepackage{iccv}
\usepackage{times}
\usepackage{epsfig}
\usepackage{graphicx}
\usepackage{amsmath}
\usepackage{amssymb}

\usepackage{algorithm}% http://ctan.org/pkg/algorithm
\PassOptionsToPackage{noend}{algpseudocode}% comment out if want end's to show
\usepackage{algpseudocode}% http://ctan.org/pkg/algorithmicx


% \newcommand*{\algrule}[1][\algorithmicindent]{\makebox[#1][l]{\hspace*{.5em}\vrule height .75\baselineskip depth .25\baselineskip}}%

% \newcount\ALG@printindent@tempcnta
% \def\ALG@printindent{%
%     \ifnum \theALG@nested>0% is there anything to print
%         \ifx\ALG@text\ALG@x@notext% is this an end group without any text?
%             % do nothing
%             \addvspace{-3pt}% FUDGE for cases where no text is shown, to make the rules line up
%         \else
%             \unskip
%             % draw a rule for each indent level
%             \ALG@printindent@tempcnta=1
%             \loop
%                 \algrule[\csname ALG@ind@\the\ALG@printindent@tempcnta\endcsname]%
%                 \advance \ALG@printindent@tempcnta 1
%             \ifnum \ALG@printindent@tempcnta<\numexpr\theALG@nested+1\relax% can't do <=, so add one to RHS and use < instead
%             \repeat
%         \fi
%     \fi
%     }%
\usepackage{etoolbox}
% the following line injects our new indent handling code in place of the default spacing
% \patchcmd{\ALG@doentity}{\noindent\hskip\ALG@tlm}{\ALG@printindent}{}{\errmessage{failed to patch}}
% \makeatother
% end vertical rule patch for algorithmicx


% Include other packages here, before hyperref.

% If you comment hyperref and then uncomment it, you should delete
% egpaper.aux before re-running latex.  (Or just hit 'q' on the first latex
% run, let it finish, and you should be clear).
\usepackage[pagebackref=true,breaklinks=true,letterpaper=true,colorlinks,bookmarks=false]{hyperref}

% \iccvfinalcopy % *** Uncomment this line for the final submission

\def\iccvPaperID{****} % *** Enter the ICCV Paper ID here
\def\httilde{\mbox{\tt\raisebox{-.5ex}{\symbol{126}}}}


\DeclareMathOperator*{\argmax}{arg\,max}
\DeclareMathOperator*{\argmin}{arg\,min}
\usepackage{mathrsfs}
\newcommand{\RED}[1]{#1}

\newcommand\TODO[1]{{\color{red}{TODO: #1}}}
\newcommand\UPDATE[1]{{\color{blue}{#1}}}
\newcommand\SOURCE[1]{{\color{green}{(from: #1)}}}
% \newcommand\CHANGED[2]{{{\color{blue}{#1}}\color{green}{#2}}}

\usepackage[dvipsnames]{xcolor}
\usepackage{enumitem}

% Pages are numbered in submission mode, and unnumbered in camera-ready
\ificcvfinal\pagestyle{empty}\fi
\begin{document}

%%%%%%%%% TITLE
\title{\LaTeX\ Author Guidelines for ICCV Proceedings}

\author{First Author\\
Institution1\\
Institution1 address\\
{\tt\small firstauthor@i1.org}
% For a paper whose authors are all at the same institution,
% omit the following lines up until the closing ``}''.
% Additional authors and addresses can be added with ``\and'',
% just like the second author.
% To save space, use either the email address or home page, not both
\and
Second Author\\
Institution2\\
First line of institution2 address\\
{\tt\small secondauthor@i2.org}
}

\maketitle
%\thispagestyle{empty}



\input{chapters/abstract.tex}

% !TEX root = ../agglo_clust_review.tex

\section{Introduction}

\begin{itemize}
\item \textbf{General intro about CNN:} big success in pixel-level image understanding tasks: boundary detection \cite{arbelaez2011contour,xie2015holistically,maninis2018convolutional}, semantic segmentation \cite{long2015fully,chen2018deeplab,kong2018recurrent}, optical flow \cite{weinzaepfel2013deepflow,dosovitskiy2015flownet}, and pose estimation \cite{wei2016convolutional,cao2017realtime}
% \SOURCE{\cite{kong2018recurrent}}
\item Many recent successful methods generate \textbf{region proposals} and classify the objects in the bounding box \cite{yang2012layered,ladicky2010and,hariharan2014simultaneous,chen2015multi,dai2016instance,liang2016reversible,he2017mask}
\item \emph{Why we do not like them}:
\begin{itemize}
\item rely on the object detector and non-maximum suppression heuristics to accurately “count” nb. of instances, 
\item difficult to train in an end-to-end manner: interface between instance segmentation and detection is non-differentiable
\item underperform in cluttered scenes (assignment carried out independently for each detection) 
\item not usable for wiry or articulated segments/objects 
\item nothing in the architecture preventing a pixel to be shared between two instances
\item number of instances is limited by nb. proposals processed by CNN (usually hundreds). 
% \item (architecture is complex and hard to tune and “debug”...?)
\end{itemize}

\item Among successful \textbf{proposal-free methods} there are: predict pixel embedding vectors \cite{kong2018recurrent,fathi2017semantic,newell2017associative,de2017semantic}; predict short- and long-range affinities \cite{liu2018affinity,wolf2018mutex,xie2015holistically}

\item In both approaches we can obtain the final segmentation by using a \textbf{graph clustering algorithm}
\item Recent work (\cite{wolf2018mutex}, more?) shows that it is better to use \textbf{repulsion and attraction} (directly predict it with the classifier, no need to define seeds or to fix a threshold given an hierarchy of clusters)
\item solving correlation clustering problem (multicut) is too expensive for this application (even using recently proposed heuristics)
\item our contributions:
\begin{itemize}
\item we propose a unified and simple formalization of Agglomerative Clustering and show how many of the recently proposed methods can be seen as special cases
\item we propose several new clustering algorithms, among which one proved to be robust in our experiments (average linkage with must-not-link-edges)
\item we compare different types of agglomeration clustering on a pixel grid graph with short and long-range connections, focusing on aspects like efficieny, robustness (and energy) 

\end{itemize}
\item on other datasets (connectomics) many methods are based on multi-step pipelines first predicting superpixels
\item \textit{if we get decent scores on CREMI}: we show that also on neuro-data it is worth to skip the hand-crafted superpixel step and compute the final segmentation directly from the CNN predictions (MWS already showed it on ISBI)

\end{itemize}


\section{Related work}
\begin{itemize}
\item \textit{Optional:} more about \textbf{Mask R-CNN}? (we actually already explain in the introduction why we do not care about it)
\item \textbf{Proposal-free instance segmentation} methods:
\begin{itemize}
\item \textbf{Predict pixel embedding vectors:} 
\begin{itemize}
\item For each pixel in the image, a CNN predicts an embedding vector, such that pixels in the same instance are represented by the same vector. (\textit{by training a model that labels pixels with unit-length vectors that live in some fixed dimension embedding space} \SOURCE{Rec. embeddings} ). \textit{With instance embedding, each object is assigned a “color” in a n-dimensional space. The network processes the image and produces a dense output, same size as the input image. Each pixel in the output of the network is a point in the embedding space. Pixels that belong to the same object are close in the embedding space while pixels that belong to different objects are distant in the embedding space. Parsing the image embedding space involves some sort of clustering algorithm.} \SOURCE{online} 
\item unsupervised or supervised; different types of loss \TODO{Find good summary}
\item clustering algorithms like DBSCAN, mean-shift or similar ones working directly in the embedding space  \TODO{Add refs, discuss with Roman}
\item \textit{Side comment}: Even with pixel embedding vectors we can deduce affinities or signed costs
\end{itemize}
\item \textbf{Other methods}: recurrent CNN, deep watershed, InstanceCut, Deep Coloring 2018 \TODO{Add more, discuss with Roman}
\item \textbf{Predicting long-range affinities:} superhuman, MWS, graph-merge on cityscapes, holistically edge detection (see MWS paper for a good review) \TODO{Add more}
\end{itemize}

\item So we need a \textbf{graph clustering algorithm}:
\begin{itemize}
\item Mention some seeded method?
\item \textit{Unsigned clustering methods:} Felzenszwalb Efficient Graph-based Image segmentation, (spectral clustering, graph cuts, not sure...)  \TODO{Here I need some input about recent stuff}
\item \textit{Merge-tree methods:} 
\begin{itemize}
\item Hierarchical clustering, ultra-metric contour map...? \TODO{This needs more research}
\item Only one from pixels I know: GMIS, hierarchical-Felzenszwalb (??) 
\item most of the others from superpixels
\item In connectomics: GALA, merge-mistakes (another section for splitting option?)
\item \textit{Similar to the boundary detection/region segmentation pipeline for natural image segmentation [6,7,8,9], most recent EM image segmentation methods use a membrane detection/cell segmentation pipeline. First, a membrane detector generates pixel-wise confidence maps of membrane predictions using local image cues [10,11,12]. Next, region-based methods are applied to transforming the membrane confidence maps into cell segments. It has been shown that region-based methods are necessary for improving the segmentation accuracy from membrane detections for EM images [13]. A common approach to region-based segmentation is to transform a membrane confidence map into over-segmenting superpixels and use them as “building blocks” for final segmentation. To correctly combine superpixels, greedy region agglomeration based on certain boundary saliency has been shown to work [14]. Meanwhile, structures, such as loopy graphs [15,16] or trees [17,18,19], are more often imposed to represent the region merging hierarchy and help transform the superpixel combination search into graph labeling problems. To this end, local [17,16] or structured [18,19] learning based methods are developed.} \SOURCE{SSHMT} \SOURCE{Jan Funke Multicut proposals}
\end{itemize}
\item \textbf{Signed graph}
\begin{itemize}
\item Multicut pipiline and heuristics
\item Jan Funke proposals
\item \textit{Cannot-link constraints:} initially introduced as hard-contraints, then relaxed (MWS, Greedy Fixation)
\item Local-search approximations of MC: GAEC, GreedyFixation
\end{itemize}
\end{itemize}

\end{itemize}




\begin{algorithm}
  \caption{Graph Agglomerative Clustering}
\setlength{\parindent}{\algorithmicindent} \textbf{Inputs:}
     \begin{itemize}[leftmargin=1.3cm,topsep=0.1pt,itemsep=-1.ex]
    %  \setlength\itemsep{0.em}
   \item $\mathcal{G}(V,E)$ with $|V|=N$, $|E|=M$
   \item signed edge weights $w:\,E\rightarrow\mathbb{R}$
   \item {\color{blue}addMustNotLink} $\in \{ True, False\}$
   \end{itemize}
   \vspace{0.4em}
   
\setlength{\parindent}{\algorithmicindent} \textbf{Output:} Final clustering $\Pi$

%   \hspace*{\algorithmicindent} \textbf{Inputs:} $\mathcal{G}(V,E)$ with signed costs $w:\,E\rightarrow\mathbb{R}$. \\
%   Prova \\
%   \hspace*{\algorithmicindent} \textbf{Outputs:} Final clustering $\Pi$\\

  \hspace*{\algorithmicindent} 
  \begin{algorithmic}[1]


    % \Procedure{GraphEdgeContr}{{\color{blue}bool \emph{addConstraints}}}
      % \State $\mathcal{G}'\gets \mathcal{G}(V,E^+ \cup E^-)$ \Comment{Initialize the contracted graph}
      \State $\mathcal{G}'(V', E') \gets \mathcal{G}(V, E)$  \Comment{Contracted graph}
    %   \State PQ $\gets$ Sort $e\in E$ in descending order of $|w_e|$
        \State PQ.push$(|w_e|, w_e, e) \quad \forall e \in E $  \Comment{Sort edges by $|w_e|$}
      
      \State $\Pi \gets \{ \{v_1\}, ..., \{v_N\} \}$ \Comment{Initial clustering}
      \State $E_\dagger \gets \{\}$ \Comment{Set of must-not-link edges}
    %   \State PQ.push$(e,  ) \quad \forall e \in E $  
    \State
      \While{PQ is \textbf{not} empty}
        \State $|\tilde{w}|, \tilde{w}, e_{uv} \gets $ PQ.popHighest()
        \If{ $e_{uv} \notin E' $} 
            \State \textbf{continue}
        \EndIf
        \If{({\color{ForestGreen}\textbf{$\tilde{w} > 0$}}) \textbf{and} ($e_{uv} \notin E_\dagger$)}
        %   \State $u,v \gets u,v \in V' : $
        %   \State $S_u \gets S \in \Pi$ : $ u \in S$
        %   \State $S_v \gets S \in \Pi$ : $ v \in S$
          \State PQ, $\,E_\dagger,\,\, E' \gets$ \textsc{deleteDoubleEdges}($u,v$)
        %   \State mergeDoubleEdges($u,v$) \Comment{Update PQ, $E_\dagger, \mathcal{G}'$}
          
        %   \State Update costs of double edges;
        %   \State Propagate constrained flags of double edges;
          \State $V' \gets V' \setminus \{ v\}$, $\quad E' \gets E' \setminus \{ e_{uv}\}$
        %   \State $ S_u \gets S_u \cup S_v$
          \State $\Pi \gets \Pi \cup \{ S_u^\Pi \cup S_v^\Pi \} \setminus \{ S_u^\Pi, S_v^\Pi \}$
          % \For{every new double edge}
          %   \State Delete double edges
          %   \State Insert new one with updated cost
          % \EndFor
        \EndIf
        \If{({\color{red}\textbf{$\tilde{w} \leq 0$}}) \textbf{and} {\color{blue}addMustNotLink}}
          \State $ E_\dagger \gets E_\dagger \cup \{e_{uv} \} $
        \EndIf
      \EndWhile
      \State
    %   \State
      \Return $\Pi$
      % \State
    % \EndProcedure

  \end{algorithmic}
  \hspace*{2cm} 
    \begin{algorithmic}[1]

    \Function{DeleteDoubleEdges}{$u,v$}
      % \State $\mathcal{G}'\gets \mathcal{G}(V,E^+ \cup E^-)$ \Comment{Initialize the contracted graph}
      \State $\mathcal{N}_u = \{ t \in V' | e_{ut}\in E'  \}$
      \State $\mathcal{N}_v = \{ t \in V' | e_{vt}\in E'  \}$
      \For{$t \in \mathcal{N}_u  \cap \mathcal{N}_v$ }
        \State $|\tilde{w}_1|, \tilde{w}_1, e_1 \gets $ PQ.pop($e_{ut}$)
        \State $|\tilde{w}_2|, \tilde{w}_2, e_2 \gets $ PQ.pop($e_{vt}$)
        % \State $e_1, e_2 \gets e_{ut}, e_{vt}$
        \State $E' \gets E' \setminus \{ e_2\}$ %\Comment{Delete double edge}
        \If{$e_2 \in E_\dagger $} \Comment{Propagate must-not-link}
            \State $ E_\dagger \gets E_\dagger \cup \{e_1 \} $
        \EndIf
        % \State $\tilde{w}_1, \tilde{w}_2 \gets $ PQ.pop($e_1$), PQ.pop($e_2$)
        \State $\tilde{w}_{\mathrm{new}} \gets$ \textsc{linkageCriteria}$(\tilde{w}_1, \tilde{w}_2)$
        \State PQ.push($|\tilde{w}_{\mathrm{new}}|$, $\tilde{w}_{\mathrm{new}}$,  $e_1$)
        
        
      \EndFor
      
      \State
    %   \State
      \Return PQ, $E_\dagger, E'$
    %   % \State


    \EndFunction

  \end{algorithmic}
  
\end{algorithm}



\section{Grand Unified Whatever Clustering}
\begin{itemize}
\item Define graph formalism and clustering $\Pi$
\item \TODO{Description of the algorithm}
\item Linkage criteria and table:
\begin{itemize}
\item Mutex Watershed (MWS), [proof of equivalence included in Appendix]
\item Greedy Additive Edge Contraction (GAEC), 
\item Greedy Fixation (GF), 
\item Hierarchical Agglomerative Clustering with single, complete, average (UPGMA) and weighted average (WPGMA) linkage criteria
\item GALA (learned update rule)
\item Something depending on the node size...?
\end{itemize}
\item Pseudocode does not (exactly) include more general options: size of the nodes, features for CNN, histogram for median \TODO{But I think I should keep it simple}
\item Comment about must-not-link relations: they give high proprity to the most confident repulsive edges \item I don't think I need to mention the merge tree (although it could be easily deduced from the code, as a sequence of clustering)
\item Define multicut energy (or move to following section)
\end{itemize}


\section{Experiments}
\begin{itemize}
\item Nevertheless, in both approaches we can define a grid graph (each vertex representing a pixel) with signed weights and find the final instance segmentation by using a graph partitioning algorithm:
\begin{itemize}
\item In Method 1, the graph is complete and the edge weights represent similarities between pairs of pixel embedding vectors. In most proposed approaches, embedding vectors representing distinct instances should be "distant enough" in the embedding space (a minimum distance is usually used during the training of the CNN). Thus, this threshold distance can be used to define signed affinities, representing attraction or repulsion between pairs of pixels;
\item  In Method 2, the short- and long-range affinities predicted by the CNN are directly used as signed edge weights in the grid graph
\end{itemize}
\end{itemize}

\subsection{Results summary}
\begin{itemize}
\item in the proposed simple formalism for Signed Graph Edge Contraction Clustering, we also include a modified version of Hierarchical agglomerative clustering with cannot-link constraints that was not mentioned in literature so far and in our experiments proved to be quite robust to noise and outliers
\item we compare different types of Signed Graph Edge Contraction Clustering, showing in which situations is preferable to use one or the other and which one can make a better use of the long-range CNN predictions; 
\item we focus our comparison on efficiency and robustness to mistakes in the CNN predictions 
\item in general, our experiments show that long-range predictions should be used not only during the CNN training, but also as an input of the clustering algorithm

\end{itemize}

{\small
\bibliographystyle{ieee}
\bibliography{agglo_clust}
}

\section{Supplementary Material}
\begin{itemize}
\item Equivalence Mutex Watershed and Greedy Edge Contraction
\end{itemize}


\end{document}
