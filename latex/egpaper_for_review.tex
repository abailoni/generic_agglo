\documentclass[10pt,twocolumn,letterpaper]{article}

\usepackage{iccv}
\usepackage{times}
\usepackage{epsfig}
\usepackage{graphicx}
\usepackage{amsmath}
\usepackage{amssymb}

% Include other packages here, before hyperref.

% If you comment hyperref and then uncomment it, you should delete
% egpaper.aux before re-running latex.  (Or just hit 'q' on the first latex
% run, let it finish, and you should be clear).
\usepackage[pagebackref=true,breaklinks=true,letterpaper=true,colorlinks,bookmarks=false]{hyperref}

% \iccvfinalcopy % *** Uncomment this line for the final submission

\def\iccvPaperID{****} % *** Enter the ICCV Paper ID here
\def\httilde{\mbox{\tt\raisebox{-.5ex}{\symbol{126}}}}


\DeclareMathOperator*{\argmax}{arg\,max}
\DeclareMathOperator*{\argmin}{arg\,min}
\usepackage{mathrsfs}
\newcommand{\RED}[1]{#1}

\newcommand\TODO[1]{{\color{red}{TODO: #1}}}
\newcommand\UPDATE[1]{{\color{blue}{#1}}}
\newcommand\SOURCE[1]{{\color{green}{(from: #1)}}}
% \newcommand\CHANGED[2]{{{\color{blue}{#1}}\color{green}{#2}}}



% Pages are numbered in submission mode, and unnumbered in camera-ready
\ificcvfinal\pagestyle{empty}\fi
\begin{document}

%%%%%%%%% TITLE
\title{\LaTeX\ Author Guidelines for ICCV Proceedings}

\author{First Author\\
Institution1\\
Institution1 address\\
{\tt\small firstauthor@i1.org}
% For a paper whose authors are all at the same institution,
% omit the following lines up until the closing ``}''.
% Additional authors and addresses can be added with ``\and'',
% just like the second author.
% To save space, use either the email address or home page, not both
\and
Second Author\\
Institution2\\
First line of institution2 address\\
{\tt\small secondauthor@i2.org}
}

\maketitle
%\thispagestyle{empty}


%%%%%%%%% ABSTRACT
\begin{abstract}
   The ABSTRACT is to be in fully-justified italicized text, at the top
   of the left-hand column, below the author and affiliation
   information. Use the word ``Abstract'' as the title, in 12-point
   Times, boldface type, centered relative to the column, initially
   capitalized. The abstract is to be in 10-point, single-spaced type.
   Leave two blank lines after the Abstract, then begin the main text.
   Look at previous ICCV abstracts to get a feel for style and length.
\end{abstract}

%%%%%%%%% BODY TEXT
\section{Introduction}

\begin{itemize}
\item \textbf{Intro about CNN:} \textit{The successes of deep convolutional neural nets (CNNs) at image classification has spawned a flurry of work in computer vision on adapting these models to pixel-level image understanding tasks, such as boundary detection [1, 90, 64], semantic segmentation [60, 10, 46], optical flow [87, 20], and pose estimation [85, 7].} \SOURCE{Rec. embeddings}
\item Intro about instance segmentation
\item \textbf{Mask R-CNN:} why we do not like it 
\begin{itemize}
\item \textit{most recent successful approaches to instance segmentation have adopted more heuristic approaches that first use an object detector to enumerate candidate instances and then perform pixel-level segmentation of each instance [57, 17, 55, 56, 2]. Alternately one can generate generic proposal segments and then label each one with a semantic detector [31, 12, 32, 16, 82, 34]. In either case the detection and segmentation steps can both be mapped to standard binary classification losses. While effective, these approaches are somewhat unsatisfying since: (1) they rely on the object detector and non-maximum suppression heuristics to accurately “count” the number of instances, (2) they are difficult to train in an end-to-end manner since the interface between instance segmentation and detection is non-differentiable, and (3) they underperform in cluttered scenes as the assignment of pixels to detections is carried out independently for each detection 1.} \SOURCE{Rec. embeddings}
\item elongated segments
\item \textit{There are three fundamental flaws in a proposal-based instance segmentation architecture. First, two objects may share the same bounding box, or a very similar boxes. In this case, the mask head, has no way of telling which object to pick in the box. This is a serious problem with stringy like object that have low fill rate in their bounding box (e.g. bicycles and chairs). Second, there is nothing in the architecture preventing a pixel to be shared between two instance. Third, the number of instances is limited by the number of proposals processed by the network (usually hundreds). More over, the architecture is complex and hard to tune and “debug”. In object detection, a precursor to this problem, there are already successes being made to use a simpler, single-stage, architectures e.g. RetinaNet.}
\end{itemize}
\item More intro about proposal-free instance segmentation? Perhaps not needed

\item One common method: \textbf{predict pixel embedding vectors.} For each pixel in the image, a CNN predicts an embedding vector, such that pixels in the same instance are represented by the same vector. (\textit{by training a model that labels pixels with unit-length vectors that live in some fixed dimension embedding space} \SOURCE{Rec. embeddings} ). \textit{With instance embedding, each object is assigned a “color” in a n-dimensional space. The network processes the image and produces a dense output, same size as the input image. Each pixel in the output of the network is a point in the embedding space. Pixels that belong to the same object are close in the embedding space while pixels that belong to different objects are distant in the embedding space. Parsing the image embedding space involves some sort of clustering algorithm.} \SOURCE{online} 
\item use clustering algorithms like DBSCAN, mean-shift or similar ones working directly in the embedding space. \UPDATE{More clustering options..?}
\item Other methods directly \textbf{Predict short and long-range affinities}: for each pixel we fix a set of neighboring pixels (not necessarily limited to the direct neighbors);  a CNN predicts then affinities, that represent how likely it is for each pair of neighboring pixels to be in the same instance.
\item Even with pixel embedding vectors we can deduce affinities or signed costs
\item It was shown that it is good to describe the problem in terms of a signed graph, or better: pixel-grid graph with short- and long-range connections; segmentation given by (signed) graph partitioning algorithm  
\item makes sense to use repulsion/attraction; directly predicted by CNN; no need to fix threshold/number of seeds/etc...
\item \textit{Multicut has the great advantage that a “natural” partitioning of a graph can be found, without needing to specify a desired number of clusters, or a termination criterion, or one seed per region. Its great drawback is that its optimization is NP-hard.} \SOURCE{MWS}. (too big even for multicut approximations)
\item Efficient alternatives proposed independently: local-search approximations, MWS, graph-merge with average \TODO{Expand this part (add more items to the list) and highlight the fact that so far a clear comparison was missing and everyone was using something kind of random}
\item To summarize our contributions:
\begin{itemize}
\item we propose a unified and simple formalization of Signed Graph Edge Contraction Clustering and show how many of the recently proposed methods can be seen as special cases
\item we compare different types of agglomeration clustering on a pixel grid graph with short and long-range connections, focusing on aspects like efficieny, robustness (and energy) 
\item we propose a new a constrained version of average agglomerative clustering that proved to be quite robust in most of our experiments
\end{itemize}
\end{itemize}


\section{Related work}
\begin{itemize}
\item \textbf{Pixel embedding vectors:} unsupervised or supervised; different types of loss \TODO{Find good summary}
\item clustering algorithms like DBSCAN, mean-shift orsimilar ones working directly in the embedding space  \TODO{Add refs, discuss with Roman}
\item \textbf{Predicting affinities (and long-range):} superhuman, MWS, graph-merge on cityscapes, holistically edge detection (see MWS paper for a good review) \TODO{Add more}
\item \textbf{Pipelines in connectomics (or better: merge trees):} \textit{Similar to the boundary detection/region segmentation pipeline for natural image segmentation [6,7,8,9], most recent EM image segmentation methods use a membrane detection/cell segmentation pipeline. First, a membrane detector generates pixel-wise confidence maps of membrane predictions using local image cues [10,11,12]. Next, region-based methods are applied to transforming the membrane confidence maps into cell segments. It has been shown that region-based methods are necessary for improving the segmentation accuracy from membrane detections for EM images [13]. A common approach to region-based segmentation is to transform a membrane confidence map into over-segmenting superpixels and use them as “building blocks” for final segmentation. To correctly combine superpixels, greedy region agglomeration based on certain boundary saliency has been shown to work [14]. Meanwhile, structures, such as loopy graphs [15,16] or trees [17,18,19], are more often imposed to represent the region merging hierarchy and help transform the superpixel combination search into graph labeling problems. To this end, local [17,16] or structured [18,19] learning based methods are developed.} \SOURCE{SSHMT} \TODO{This clearly need to be extended and reformulated}
\item \textbf{Cannot-link constraints:} initially introduced as hard-contraints, then relaxed (MWS, Greedy Fixation)
\item \textbf{Unsigned clustering methods:} Felzenszwalb Efficient Graph-based Image segmentation, \TODO{Hierarchical clustering \& Co.}
\end{itemize}


\section{Grand Unified Edge Contraction Algorithm}
\begin{itemize}
\item Rows in the table:
\begin{itemize}
\item Mutex Watershed (MWS), [proof of equivalence included in Appendix]
\item Greedy Additive Edge Contraction (GAEC), 
\item Greedy Fixation (GF), 
\item Hierarchical Agglomerative Clustering with single, complete, average (UPGMA) and weighted average (WPGMA) linkage criteria
\item GALA (learned update rule)
\item Something depending on the node size...?
\end{itemize}
\end{itemize}


\section{Experiments}
\begin{itemize}
\item Nevertheless, in both approaches we can define a grid graph (each vertex representing a pixel) with signed weights and find the final instance segmentation by using a graph partitioning algorithm:
\begin{itemize}
\item In Method 1, the graph is complete and the edge weights represent similarities between pairs of pixel embedding vectors. In most proposed approaches, embedding vectors representing distinct instances should be "distant enough" in the embedding space (a minimum distance is usually used during the training of the CNN). Thus, this threshold distance can be used to define signed affinities, representing attraction or repulsion between pairs of pixels;
\item  In Method 2, the short- and long-range affinities predicted by the CNN are directly used as signed edge weights in the grid graph
\end{itemize}
\end{itemize}

\subsection{Results summary}
\begin{itemize}
\item in the proposed simple formalism for Signed Graph Edge Contraction Clustering, we also include a modified version of Hierarchical agglomerative clustering with cannot-link constraints that was not mentioned in literature so far and in our experiments proved to be quite robust to noise and outliers
\item we compare different types of Signed Graph Edge Contraction Clustering, showing in which situations is preferable to use one or the other and which one can make a better use of the long-range CNN predictions; 
\item we focus our comparison on efficiency and robustness to mistakes in the CNN predictions 
\item in general, our experiments show that long-range predictions should be used not only during the CNN training, but also as an input of the clustering algorithm

\end{itemize}

{\small
\bibliographystyle{ieee}
\bibliography{egbib}
}

\section{Supplementary Material}
\begin{itemize}
\item Equivalence Mutex Watershed and Greedy Edge Contraction
\end{itemize}

%-------------------------------------------------------------------------

% \section{ICCV stuff}
% \subsection{Language}

% All manuscripts must be in English.

% \subsection{Dual submission}

% Please refer to the author guidelines on the ICCV 2019 web page for a
% discussion of the policy on dual submissions.

% \subsection{Paper length}
% Papers, excluding the references section,
% must be no longer than eight pages in length. The references section
% will not be included in the page count, and there is no limit on the
% length of the references section. For example, a paper of eight pages
% with two pages of references would have a total length of 10 pages.

% Overlength papers will simply not be reviewed.  This includes papers
% where the margins and formatting are deemed to have been significantly
% altered from those laid down by this style guide.  Note that this
% \LaTeX\ guide already sets figure captions and references in a smaller font.
% The reason such papers will not be reviewed is that there is no provision for
% supervised revisions of manuscripts.  The reviewing process cannot determine
% the suitability of the paper for presentation in eight pages if it is
% reviewed in eleven.  

% %-------------------------------------------------------------------------
% \subsection{The ruler}
% The \LaTeX\ style defines a printed ruler which should be present in the
% version submitted for review.  The ruler is provided in order that
% reviewers may comment on particular lines in the paper without
% circumlocution.  If you are preparing a document using a non-\LaTeX\
% document preparation system, please arrange for an equivalent ruler to
% appear on the final output pages.  The presence or absence of the ruler
% should not change the appearance of any other content on the page.  The
% camera ready copy should not contain a ruler. (\LaTeX\ users may uncomment
% the \verb'\iccvfinalcopy' command in the document preamble.)  Reviewers:
% note that the ruler measurements do not align well with lines in the paper
% --- this turns out to be very difficult to do well when the paper contains
% many figures and equations, and, when done, looks ugly.  Just use fractional
% references (e.g.\ this line is $095.5$), although in most cases one would
% expect that the approximate location will be adequate.

% \subsection{Mathematics}

% Please number all of your sections and displayed equations.  It is
% important for readers to be able to refer to any particular equation.  Just
% because you didn't refer to it in the text doesn't mean some future reader
% might not need to refer to it.  It is cumbersome to have to use
% circumlocutions like ``the equation second from the top of page 3 column
% 1''.  (Note that the ruler will not be present in the final copy, so is not
% an alternative to equation numbers).  All authors will benefit from reading
% Mermin's description of how to write mathematics:
% \url{http://www.pamitc.org/documents/mermin.pdf}.


% \subsection{Blind review}

% Many authors misunderstand the concept of anonymizing for blind
% review.  Blind review does not mean that one must remove
% citations to one's own work---in fact it is often impossible to
% review a paper unless the previous citations are known and
% available.

% Blind review means that you do not use the words ``my'' or ``our''
% when citing previous work.  That is all.  (But see below for
% techreports.)

% Saying ``this builds on the work of Lucy Smith [1]'' does not say
% that you are Lucy Smith; it says that you are building on her
% work.  If you are Smith and Jones, do not say ``as we show in
% [7]'', say ``as Smith and Jones show in [7]'' and at the end of the
% paper, include reference 7 as you would any other cited work.

% An example of a bad paper just asking to be rejected:
% \begin{quote}
% \begin{center}
%     An analysis of the frobnicatable foo filter.
% \end{center}

%   In this paper we present a performance analysis of our
%   previous paper [1], and show it to be inferior to all
%   previously known methods.  Why the previous paper was
%   accepted without this analysis is beyond me.

%   [1] Removed for blind review
% \end{quote}


% An example of an acceptable paper:

% \begin{quote}
% \begin{center}
%      An analysis of the frobnicatable foo filter.
% \end{center}

%   In this paper we present a performance analysis of the
%   paper of Smith \etal [1], and show it to be inferior to
%   all previously known methods.  Why the previous paper
%   was accepted without this analysis is beyond me.

%   [1] Smith, L and Jones, C. ``The frobnicatable foo
%   filter, a fundamental contribution to human knowledge''.
%   Nature 381(12), 1-213.
% \end{quote}

% If you are making a submission to another conference at the same time,
% which covers similar or overlapping material, you may need to refer to that
% submission in order to explain the differences, just as you would if you
% had previously published related work.  In such cases, include the
% anonymized parallel submission~\cite{Authors14} as additional material and
% cite it as
% \begin{quote}
% [1] Authors. ``The frobnicatable foo filter'', F\&G 2014 Submission ID 324,
% Supplied as additional material {\tt fg324.pdf}.
% \end{quote}

% Finally, you may feel you need to tell the reader that more details can be
% found elsewhere, and refer them to a technical report.  For conference
% submissions, the paper must stand on its own, and not {\em require} the
% reviewer to go to a techreport for further details.  Thus, you may say in
% the body of the paper ``further details may be found
% in~\cite{Authors14b}''.  Then submit the techreport as additional material.
% Again, you may not assume the reviewers will read this material.

% Sometimes your paper is about a problem which you tested using a tool which
% is widely known to be restricted to a single institution.  For example,
% let's say it's 1969, you have solved a key problem on the Apollo lander,
% and you believe that the ICCV70 audience would like to hear about your
% solution.  The work is a development of your celebrated 1968 paper entitled
% ``Zero-g frobnication: How being the only people in the world with access to
% the Apollo lander source code makes us a wow at parties'', by Zeus \etal.

% You can handle this paper like any other.  Don't write ``We show how to
% improve our previous work [Anonymous, 1968].  This time we tested the
% algorithm on a lunar lander [name of lander removed for blind review]''.
% That would be silly, and would immediately identify the authors. Instead
% write the following:
% \begin{quotation}
% \noindent
%   We describe a system for zero-g frobnication.  This
%   system is new because it handles the following cases:
%   A, B.  Previous systems [Zeus et al. 1968] didn't
%   handle case B properly.  Ours handles it by including
%   a foo term in the bar integral.

%   ...

%   The proposed system was integrated with the Apollo
%   lunar lander, and went all the way to the moon, don't
%   you know.  It displayed the following behaviours
%   which show how well we solved cases A and B: ...
% \end{quotation}
% As you can see, the above text follows standard scientific convention,
% reads better than the first version, and does not explicitly name you as
% the authors.  A reviewer might think it likely that the new paper was
% written by Zeus \etal, but cannot make any decision based on that guess.
% He or she would have to be sure that no other authors could have been
% contracted to solve problem B.

% \noindent
% FAQ\medskip\\
% {\bf Q:} Are acknowledgements OK?\\
% {\bf A:} No.  Leave them for the final copy.\medskip\\
% {\bf Q:} How do I cite my results reported in open challenges?
% {\bf A:} To conform with the double blind review policy, you can report results of other challenge participants together with your results in your paper. For your results, however, you should not identify yourself and should not mention your participation in the challenge. Instead present your results referring to the method proposed in your paper and draw conclusions based on the experimental comparison to other results.\medskip\\


% \begin{figure}[t]
% \begin{center}
% \fbox{\rule{0pt}{2in} \rule{0.9\linewidth}{0pt}}
%   %\includegraphics[width=0.8\linewidth]{egfigure.eps}
% \end{center}
%   \caption{Example of caption.  It is set in Roman so that mathematics
%   (always set in Roman: $B \sin A = A \sin B$) may be included without an
%   ugly clash.}
% \label{fig:long}
% \label{fig:onecol}
% \end{figure}

% \subsection{Miscellaneous}

% \noindent
% Compare the following:\\
% \begin{tabular}{ll}
%  \verb'$conf_a$' &  $conf_a$ \\
%  \verb'$\mathit{conf}_a$' & $\mathit{conf}_a$
% \end{tabular}\\
% See The \TeX book, p165.

% The space after \eg, meaning ``for example'', should not be a
% sentence-ending space. So \eg is correct, {\em e.g.} is not.  The provided
% \verb'\eg' macro takes care of this.

% When citing a multi-author paper, you may save space by using ``et alia'',
% shortened to ``\etal'' (not ``{\em et.\ al.}'' as ``{\em et}'' is a complete word.)
% However, use it only when there are three or more authors.  Thus, the
% following is correct: ``
%   Frobnication has been trendy lately.
%   It was introduced by Alpher~\cite{Alpher02}, and subsequently developed by
%   Alpher and Fotheringham-Smythe~\cite{Alpher03}, and Alpher \etal~\cite{Alpher04}.''

% This is incorrect: ``... subsequently developed by Alpher \etal~\cite{Alpher03} ...''
% because reference~\cite{Alpher03} has just two authors.  If you use the
% \verb'\etal' macro provided, then you need not worry about double periods
% when used at the end of a sentence as in Alpher \etal.

% For this citation style, keep multiple citations in numerical (not
% chronological) order, so prefer \cite{Alpher03,Alpher02,Authors14} to
% \cite{Alpher02,Alpher03,Authors14}.


% \begin{figure*}
% \begin{center}
% \fbox{\rule{0pt}{2in} \rule{.9\linewidth}{0pt}}
% \end{center}
%   \caption{Example of a short caption, which should be centered.}
% \label{fig:short}
% \end{figure*}

% %------------------------------------------------------------------------
% \section{Formatting your paper}

% All text must be in a two-column format. The total allowable width of the
% text area is $6\frac78$ inches (17.5 cm) wide by $8\frac78$ inches (22.54
% cm) high. Columns are to be $3\frac14$ inches (8.25 cm) wide, with a
% $\frac{5}{16}$ inch (0.8 cm) space between them. The main title (on the
% first page) should begin 1.0 inch (2.54 cm) from the top edge of the
% page. The second and following pages should begin 1.0 inch (2.54 cm) from
% the top edge. On all pages, the bottom margin should be 1-1/8 inches (2.86
% cm) from the bottom edge of the page for $8.5 \times 11$-inch paper; for A4
% paper, approximately 1-5/8 inches (4.13 cm) from the bottom edge of the
% page.

% %-------------------------------------------------------------------------
% \subsection{Margins and page numbering}

% All printed material, including text, illustrations, and charts, must be kept
% within a print area 6-7/8 inches (17.5 cm) wide by 8-7/8 inches (22.54 cm)
% high.



% %-------------------------------------------------------------------------
% \subsection{Type-style and fonts}

% Wherever Times is specified, Times Roman may also be used. If neither is
% available on your word processor, please use the font closest in
% appearance to Times to which you have access.

% MAIN TITLE. Center the title 1-3/8 inches (3.49 cm) from the top edge of
% the first page. The title should be in Times 14-point, boldface type.
% Capitalize the first letter of nouns, pronouns, verbs, adjectives, and
% adverbs; do not capitalize articles, coordinate conjunctions, or
% prepositions (unless the title begins with such a word). Leave two blank
% lines after the title.

% AUTHOR NAME(s) and AFFILIATION(s) are to be centered beneath the title
% and printed in Times 12-point, non-boldface type. This information is to
% be followed by two blank lines.

% The ABSTRACT and MAIN TEXT are to be in a two-column format.

% MAIN TEXT. Type main text in 10-point Times, single-spaced. Do NOT use
% double-spacing. All paragraphs should be indented 1 pica (approx. 1/6
% inch or 0.422 cm). Make sure your text is fully justified---that is,
% flush left and flush right. Please do not place any additional blank
% lines between paragraphs.

% Figure and table captions should be 9-point Roman type as in
% Figures~\ref{fig:onecol} and~\ref{fig:short}.  Short captions should be centred.

% \noindent Callouts should be 9-point Helvetica, non-boldface type.
% Initially capitalize only the first word of section titles and first-,
% second-, and third-order headings.

% FIRST-ORDER HEADINGS. (For example, {\large \bf 1. Introduction})
% should be Times 12-point boldface, initially capitalized, flush left,
% with one blank line before, and one blank line after.

% SECOND-ORDER HEADINGS. (For example, { \bf 1.1. Database elements})
% should be Times 11-point boldface, initially capitalized, flush left,
% with one blank line before, and one after. If you require a third-order
% heading (we discourage it), use 10-point Times, boldface, initially
% capitalized, flush left, preceded by one blank line, followed by a period
% and your text on the same line.

% %-------------------------------------------------------------------------
% \subsection{Footnotes}

% Please use footnotes\footnote {This is what a footnote looks like.  It
% often distracts the reader from the main flow of the argument.} sparingly.
% Indeed, try to avoid footnotes altogether and include necessary peripheral
% observations in
% the text (within parentheses, if you prefer, as in this sentence).  If you
% wish to use a footnote, place it at the bottom of the column on the page on
% which it is referenced. Use Times 8-point type, single-spaced.


% %-------------------------------------------------------------------------
% \subsection{References}

% List and number all bibliographical references in 9-point Times,
% single-spaced, at the end of your paper. When referenced in the text,
% enclose the citation number in square brackets, for
% example~\cite{Authors14}.  Where appropriate, include the name(s) of
% editors of referenced books.

% \begin{table}
% \begin{center}
% \begin{tabular}{|l|c|}
% \hline
% Method & Frobnability \\
% \hline\hline
% Theirs & Frumpy \\
% Yours & Frobbly \\
% Ours & Makes one's heart Frob\\
% \hline
% \end{tabular}
% \end{center}
% \caption{Results.   Ours is better.}
% \end{table}

% %-------------------------------------------------------------------------
% \subsection{Illustrations, graphs, and photographs}

% All graphics should be centered.  Please ensure that any point you wish to
% make is resolvable in a printed copy of the paper.  Resize fonts in figures
% to match the font in the body text, and choose line widths which render
% effectively in print.  Many readers (and reviewers), even of an electronic
% copy, will choose to print your paper in order to read it.  You cannot
% insist that they do otherwise, and therefore must not assume that they can
% zoom in to see tiny details on a graphic.

% When placing figures in \LaTeX, it's almost always best to use
% \verb+\includegraphics+, and to specify the  figure width as a multiple of
% the line width as in the example below
% {\small\begin{verbatim}
%   \usepackage[dvips]{graphicx} ...
%   \includegraphics[width=0.8\linewidth]
%                   {myfile.eps}
% \end{verbatim}
% }


% %-------------------------------------------------------------------------
% \subsection{Color}

% Please refer to the author guidelines on the ICCV 2019 web page for a discussion
% of the use of color in your document.

% %------------------------------------------------------------------------
% \section{Final copy}

% You must include your signed IEEE copyright release form when you submit
% your finished paper. We MUST have this form before your paper can be
% published in the proceedings.



\end{document}
