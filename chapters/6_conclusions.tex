% !TEX root = ../agglo_clust_review.tex
\section{Conclusion}
We presented both a theoretical and experimental contribution: ; why average is so cool?

We presented a generalized framework and showed how one of the new algorithms included in it, \algname{} with average linkage, represent a simple agglomerative method that does not require the user to tune parameters or generate superpixels and showed extremely competitive performances in biological images, even on the challenging and competitive task of neuron segmentation. (good-quality superpixels are strongly dataset-dependent)
Compared to previously proposed agglomerative clustering methods for signed graphs, it proved to be the one making best use of the long-range predictions of the CNN; the fact that average is superior to abs max is not surprising, given the literature about HAC with single and average linkage (due to the greediness of the maximum linkage); and enforcing cannot-link constraints, we have also seen that introducing too many long-range mainly repulsive predictions can cause strong over-clustering; tendency to under-clustering of the sum linkage can be explained with the chain-reaction effect and its tendency to grow one cluster at the time that does not allow to build good statistics between adjacent clusters. This is partially fixed by constraints, but not really...
