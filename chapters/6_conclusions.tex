% !TEX root = ../agglo_clust_review.tex
\section{Conclusion}
We have presented a novel theoretical framework for agglomerative clustering of graphs with both positive and negative weights and we have shown that several existing clustering algorithms, e.g. the Mutex Watershed \cite{wolf2018mutex}, can be reformulated as special cases of one underlying agglomerative algorithm. This framework also allowed us to introduce new algorithms, one of which, based on an \emph{Average} linkage, outperformed all the others and proved to be a simple and remarkably robust approach to process short- and long-range predictions of a CNN applied to an instance segmentation task.
On biological images, this simple average agglomeration algorithm can represent a valuable choice for an user that is not willing to spend much time tuning complex task-dependent pipelines based on superpixels.  % or test several dataset-dependent  
In future work we plan to extend the comparison to other types of graphs and explore common theoretical properties of the algorithms included in the framework.
% We presented a generalized framework and showed how one of the new algorithms included in it, \algname{} with average linkage, represent a simple agglomerative method that does not require the user to tune parameters or generate superpixels and showed extremely competitive performances in biological images, even on the challenging and competitive task of neuron segmentation. (good-quality superpixels are strongly dataset-dependent)
% Compared to previously proposed agglomerative clustering methods for signed graphs, it proved to be the one making best use of the long-range predictions of the CNN; the fact that average is superior to abs max is not surprising, given the literature about HAC with single and average linkage (due to the greediness of the maximum linkage); and enforcing cannot-link constraints, we have also seen that introducing too many long-range mainly repulsive predictions can cause strong over-clustering; tendency to under-clustering of the sum linkage can be explained with the chain-reaction effect and its tendency to grow one cluster at the time that does not allow to build good statistics between adjacent clusters. This is partially fixed by constraints, but not really...
