% !TEX root = ../agglo_clust_review.tex


\begin{figure}
        \centering
\begin{minipage}[T]{0.48\textwidth}
% \begin{table}
    \centering
    \scriptsize
    % \begin{subtable}[t!]{0.5\textwidth}\centering
        \begin{tabular}{l|l|c}
        % \toprule
        % & \multicolumn{2}{c}{\thead{Add Cannot-Link Constraints:}} \\
         % & \multicolumn{1}{c}{\thead{\textsc{No}}} & \multicolumn{1}{c}{\thead{\textsc{Yes}}} \\ \midrule
         & \algname{} Linkage & \makecell{Arand-Score}  \\ \midrule \midrule
         % & \multicolumn{1}{c}{\thead{$\beta$}}  & \thead{AP} & \multicolumn{1}{c}{\thead{$\beta$}} & \thead{AP} \\ \midrule\midrule
\textbf{\algname{}} & \textbf{Average}& \textbf{0.936 $\pm$ 0.004}  \\
Greedy Fixation \cite{levinkov2017comparative} & Sum + Constr. & 0.906 $\pm$ 0.022 \\
DTWS + LMC & -& 0.903 $\pm$ 0.016 \\
DTWS + MC & -& 0.903 $\pm$ 0.017 \\
% DTWS + avgHC &-& \\
% DTWS + MC (SR)& -& 0.900 $\pm$ 0.019 \\
% DTWS + LMC (SR)& -& 0.898 $\pm$ 0.017 \\
Mutex Watershed \cite{wolf2018mutex} & Abs. Max.  & 0.897 $\pm$ 0.012 \\
GAEC \cite{keuper2015efficient} & Sum & 0.872 $\pm$ 0.028 \\
THRESH &-& 0.221  $\pm$ 0.067 \\ 
% DTWS  & -& 0.010 $\pm$ 0.003\\
        \end{tabular}
    \captionof{table}{Arand-Scores on CREMI challenge training data}
    \label{tab:results_cremi_train}
\end{minipage}\hfill
\begin{minipage}[T]{0.48\textwidth}
    \centering
    \scriptsize
        \begin{tabular}{l|c|c}
         & \makecell{CREMI \\Score} & \makecell{Arand\\Score} \\ \midrule
Our UNet + DTWS + LMC &  \textbf{0.221} & \textbf{0.108}\\
PNI-UNet & 0.228 & 0.116 \\
Our UNet + \algname{} Avg-Linkage & 0.244 & 0.130 \\
MALA-UNet + MC \cite{funke2018large} & 0.276 & 0.132 \\
CRUNet \cite{zeng2017deepem3d} & 0.566 & 0.229 \\
LFC \cite{parag2017anisotropic} & 0.616 & \\
        \end{tabular}
    \captionof{table}{Scores on test data... \TODO{Update refs}}
    \label{tab:results_cremi_test}
\end{minipage}
\end{figure}

\subsection{CREMI Challenge} \label{sec:cremi_challenge}
We evaluate the algorithms in our framework on the competitive CREMI 2016 EM Segmentation Challenge \cite{cremiChallenge} that is currently the neuron segmentation challenge with the largest amount of training data available. The dataset comes from a serial section EM of \emph{Drosophila} fruit-fly tissue and consists of 6 volumes of 1250x1250x125 voxels at resolution 4x4x40nm, three of which present publicly available training ground truth. The results submitted to the leaderboard are evaluated using the CREMI score\footnote{\url{https://cremi.org/leaderboard/}}, based on the Adapted Rand-Score (Rand-Score) and the Variation of Information Score\cite{arganda2015crowdsourcing}. In Appendix \ref{sec:cremi_details}, we provide more details about the training of our CNN model, inspired by work of \cite{lee2017superhuman,funke2018large}.


\textbf{Additional methods tested } We compare the performances of \algname{} with other basic and state-of-the-art post-processing methods. To ensure a fair comparison, we test all methods on the same predictions of our CNN model. As basic method, we perform a simple thresholding (THRESH) by running connected components on a boundary map generated from the CNN affinities (see Appendix \ref{sec:cremi_details} for all details on this and the following methods). On the other hand, almost all state-of-the-art methods for neuron-segmentation first generates 2D superpixels and then apply a graph partitioning algorithm, as we showed in the related work section. Superpixels are usually computed with a watershed algorithm seeded at the maxima of a distance transform generated from a boundary map (WSDT). In our comparison we use approximated methods of the multicut (MC) and lifted multicut problem (LMC).


