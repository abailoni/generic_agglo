% !TEX root = ../agglo_clust_review.tex


\subsection{Data: CREMI Challenge} \label{sec:cremi_challenge}
We evaluate the algorithms in our framework on the competitive CREMI 2016 EM Segmentation Challenge \cite{cremiChallenge} that is currently the neuron segmentation challenge with the largest amount of training data available. The dataset comes from serial section EM of \emph{Drosophila} fruit-fly tissue and consists of 6 volumes of 1250x1250x125 voxels at resolution 4x4x40nm, three of which present publicly available training ground truth. The results submitted to the leaderboard are evaluated using the CREMI score\footnote{\url{https://cremi.org/leaderboard/}}, based on the Adapted Rand-Score (Rand-Score) and the Variation of Information Score \cite{arganda2015crowdsourcing}. In Appendix \ref{sec:cremi_details}, we provide more details about the training of our CNN model, inspired by work of \cite{lee2017superhuman,funke2018large}.


\textbf{Additional methods tested } We compare the performances of \algname{} with other basic and state-of-the-art post-processing methods. To ensure a fair comparison, we test all methods on the same predictions of our CNN model. As basic method, we perform a simple thresholding (THRESH) by running connected components on a boundary map generated from the CNN affinities (see Appendix \ref{sec:cremi_details} for more details on this and the following methods). On the other hand, most of state-of-the-art methods for neuron-segmentation first generates 2D superpixels and then apply a graph partitioning algorithm, since this approach so far proved to be the most reliable that could scale up to the size of the problem. Superpixels are usually computed with a watershed algorithm seeded at the maxima of a boundary distance transform (WSDT). The algorithms employed in our comparison to partition the superpixel graph were given by approximations of the multicut (WSDT+MC) and lifted multicut (WSDT+LMC) problems.


