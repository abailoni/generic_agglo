% !TEX root = ../agglo_clust_review.tex
\section{Experiments on CityScapes}\label{sec:cityscapes_exp}
We also evaluate the performances of \algname{} on the CityScapes dataset \cite{cordts2016cityscapes}, which consists of 5000 street-scene images recorded by car-mounted cameras with resolution 1024$\times$2048: 2975 images for training, 500 for validation and 1525 for testing. Objects with instance-level annotations belong to the following classes: person, rider, car, truck, bus, train, motorcycle, bicycle. For our experiments, we used the state-of-the-art proposal-free pipeline proposed in \cite{liu2018affinity} (GMIS) that achieved superior performances compared to other proposal-based methods like Mark R-CNN and employs a similar model to the one applied by us on neuron-segmentation.
Their trained model was publicly available, so we simply replaced their final graph-merging algorithm (GMIS-Agglo) with \algname{}. See Appendix \ref{sec:cityscapes_appendix} for more details on how we fine-tuned their instance branch model by using a \emph{S\o resen-Dice} loss similarly to \cite{wolf2018mutex} and obtained in this way sharper affinity-predictions.

Results are summarized in Table \ref{tab:results_cityscapes_val} and confirm our findings on neuron-segmentation: \algname{} with average linkage achieves the best scores, whereas other linkage tend to introduce more false region-splits, like \emph{Abs. Max.}, or false region-merge, like \emph{Sum}. GMIS-Agglo requires the user to tune several threshold parameters and it was probably tailored to the original affinities predicted by \cite{liu2018affinity}, so it did not generalize well to our fine-tuned model and it achieved lower scores compared to the original AP value of 34.1 in \cite{liu2018affinity}.  
